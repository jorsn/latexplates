\begingroup
\def\cmd#1{\texttt{\textbackslash{}#1}}
\begin{anfxnote}[author=\noindent JR]{template}
  The template for this document is based on
  \url{https://github.com/jorsn/latexplates}.
\end{anfxnote}

\begin{anfxwarning}[author=\noindent JR]{edit notes}
  The \textsf{fixme} package is loaded, so
  \begin{itemize}
    \item you can make editorial notes with
      \\
        \texttt{\cmd{fxnote}[author=<author>]%
      \twocol{\\\hbox{}\quad}%
          \{<margin comment>\}},
      \\
        \texttt{\cmd{fxnote*}[author=<author>]%
      \twocol{\\\hbox{}\quad}%
          \{<comment>\}\{<text to highlight>\}},

  \item
    for notes that have to be resolved before submission,
    \cmd{fxfatal} and \cmd{fxfatal*} are used like \cmd{fxnote},

  \item
    in the end of the document, there is also a \MakeLowercase\englishlistfixmename .

  \item
    The final submission should get the class option \texttt{final}. This hides
    edit notes and the list of corrections, and it produces a compile error for
    each remaining \cmd{fxfatal} calls.
\end{itemize}
\end{anfxwarning}
\endgroup

\begin{anfxnote}[author=\noindent JR]{Macro conventions}
  \section*{Macro conventions}
  \label{meta:macros}
  \raggedright
  Please do not use \verb|\mathbf|, which does not work for all (math) symbols,
  but instead use the command \verb|\bm|, which is more flexible.
  Mixing both leads to inconsistent output,
  e.g. \verb|\mathbf{X, \alpha}| $\mapsto \mathbf{X, \alpha}$,
  while \verb|\bm{X, \alpha}| $\mapsto \bm{X, \alpha}$.
  The files \verb|sty/shorthands.sty| and \verb|local-shorthands.tex|
  define many shorthands, e.g.

  \begin{multicols} 2
    \centering
    \begin{tblr}{colspec={ll},verb}
      \verb|\tup{a}|             & $\tup{a}$ \\
      \verb|\paren{a}|           & $\paren{a}$ \\
      \verb|a\intv{b}c|          & $a\intv{b}c$ \\
      \verb|a\intv{\frac 1 2}c|  & $a\intv{\frac 1 2}c$ \\
      \verb|a\intv*{\frac 1 2}c| & $a\intv*{\frac 1 2}c$ \\
      \verb|\brack[\big]{a}|     & $\brack[\big]{a}$ \\
      \verb|\abs[\Big]{a}|       & $\abs[\Big]{a}$ \\
      \verb|\card[\bigg]{\cX_1}| & $\card[\bigg]{\cX_1}$ \\
      \verb|\set[\Bigg]{a}|      & $\set[\Bigg]{a}$ \\
    \end{tblr}

    \begin{tblr}{colspec={ll},verb}
      \verb|\bX|     & $\bX$ \\
      \verb|\bbx|    & $\bbx$ \\
      \verb|\cX|     & $\cX$ \\
      \verb|\bcX|    & $\bcX$ \\
      \verb|\bbN|    & $\bbN$ \\
      \verb|\bbR|    & $\bbR$ \\
      \verb|\expect| & $\expect$ \\
      \verb|\One|    & $\One$ \\
      \verb|\ind{e^{A^2} \le 1}| & $\ind{e^{A^2} \le 1}$ \\
      \verb|\ind[\bigg]{a}|       & $\ind[\bigg]{a}$ \\
    \end{tblr}
  \end{multicols}

  \noindent
  Please define your own shorthands in \verb|local-shorthands.tex|.
\end{anfxnote}

% vim: ts=2 sw=2
